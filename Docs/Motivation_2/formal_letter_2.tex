%% start of file `template.tex'.
%% Copyright 2006-2013 Xavier Danaux (xdanaux@gmail.com).
%
% This work may be distributed and/or modified under the
% conditions of the LaTeX Project Public License version 1.3c,
% available at http://www.latex-project.org/lppl/.
%Version for spanish users, by dgarhdez

\documentclass[11pt,a4paper,roman]{moderncv}        % possible options include font size ('10pt', '11pt' and '12pt'), paper size ('a4paper', 'letterpaper', 'a5paper', 'legalpaper', 'executivepaper' and 'landscape') and font family ('sans' and 'roman')
%\usepackage[es-lcroman]{babel}


% moderncv themes
\moderncvstyle{classic}                            % style options are 'casual' (default), 'classic', 'oldstyle' and 'banking'
\moderncvcolor{green}                              % color options 'blue' (default), 'orange', 'green', 'red', 'purple', 'grey' and 'black'
%\renewcommand{\familydefault}{\sfdefault}         % to set the default font; use '\sfdefault' for the default sans serif font, '\rmdefault' for the default roman one, or any tex font name
%\nopagenumbers{}                                  % uncomment to suppress automatic page numbering for CVs longer than one page

% character encoding
\usepackage[T1]{fontenc}
\usepackage[utf8]{inputenc}                       % if you are not using xelatex ou lualatex, replace by the encoding you are using
%\usepackage{CJKutf8}                              % if you need to use CJK to typeset your resume in Chinese, Japanese or Korean

% adjust the page margins
\usepackage[scale=0.75]{geometry}
%\setlength{\hintscolumnwidth}{3cm}                % if you want to change the width of the column with the dates
%\setlength{\makecvtitlenamewidth}{10cm}           % for the 'classic' style, if you want to force the width allocated to your name and avoid line breaks. be careful though, the length is normally calculated to avoid any overlap with your personal info; use this at your own typographical risks...

%---------------------------------------------------------------------------
% Font and Typefaces
%---------------------------------------------------------------------------
\usepackage[default, light, semibold]{sourcesanspro}
%\usepackage{calligra}

% personal data
\name{Yernur}{Baibolatov}
\title{Motivation letter}                               % optional, remove / comment the line if not wanted
\address{Brusilovsky str. 159}{050000 Almaty}{Kazakhstan}% optional, remove / comment the line if not wanted; the "postcode city" and and "country" arguments can be omitted or provided empty
\phone[mobile]{+7-707-463-2528}                   % optional, remove / comment the line if not wanted
%\phone[fixed]{+2~(345)~678~901}                    % optional, remove / comment the line if not wanted
%\phone[fax]{+3~(456)~789~012}                      % optional, remove / comment the line if not wanted
\email{yernurb@gmail.com}                               % optional, remove / comment the line if not wanted
%\homepage{www.johndoe.com}                         % optional, remove / comment the line if not wanted
%\extrainfo{additional information}                 % optional, remove / comment the line if not wanted
%\photo[64pt][0.4pt]{picture}                       % optional, remove / comment the line if not wanted; '64pt' is the height the picture must be resized to, 0.4pt is the thickness of the frame around it (put it to 0pt for no frame) and 'picture' is the name of the picture file
%\quote{Some quote}                                 % optional, remove / comment the line if not wanted

% to show numerical labels in the bibliography (default is to show no labels); only useful if you make citations in your resume
%\makeatletter
%\renewcommand*{\bibliographyitemlabel}{\@biblabel{\arabic{enumiv}}}
%\makeatother
%\renewcommand*{\bibliographyitemlabel}{[\arabic{enumiv}]}% CONSIDER REPLACING THE ABOVE BY THIS

% bibliography with mutiple entries
%\usepackage{multibib}
%\newcites{book,misc}{{Books},{Others}}
%----------------------------------------------------------------------------------
%            content
%----------------------------------------------------------------------------------
\begin{document}
%-----       letter       ---------------------------------------------------------
% recipient data
\recipient{Claudia Wendt}{Potsdam Graduate School \\ Am Kanal 47 \\ 14467 Potsdam}
\date{\today}
\opening{Dear expert committee,}
\closing{Yours sincerely}
%\enclosure[Adjunto]{CV}          % use an optional argument to use a string other than "Enclosure", or redefine \enclname
\makelettertitle

with this letter, I would like to express my interest in applying for the completion scholarship program,
in order to finish my Ph.D. in physics at the University of Potsdam. Since my very childhood,
I’ve always dreamed of becoming a physicist, and pursue a scientific career.

To achieve this goal, I studied physics at the Kazakh National University,
the best option I had available to me at the time. After graduation with excellence, 
I was fortunate enough to get acquainted with Prof. Arkadi Pikovsky and Prof. Michael Rosenblum and
had a chance to work with them at the University of Potsdam.

It was a fruitful collaboration where I studied the behavior of oscillator ensembles with nonlinear coupling
and as a result, we published two articles together. However, during these years of common work,
I understood that I still need to learn a lot.

That is where I got introduced to Prof. Frank Spahn. I was always interested in two fields, cosmic exploration,
and systems with a huge number of interacting objects. These two fields perfectly merged in the topic of research of
Prof. Frank Spahn, planetary rings, and cosmic dust research. He was very kind to accept me as a part of their group,
where I started researching new and fascinating subjects, such as kinetics, granular gases, and orbital motion of bodies.
It was a long journey, but finally, after accumulating a certain level of knowledge, with the help of Prof. Frank Spahn
and his group members, it became possible to develop some of my ideas into a publishable state. 

This is why I am writing this letter, to ask the expert committee to give me a chance to finish the work
I have done with Prof. Frank Spahn, publish the results and write my Ph.D. thesis.

\vspace{0.5cm}


\makeletterclosing

\end{document}


%% end of file `template.tex'.
